\def\mytitle{ARM ASSIGNMENT}
\def\mykeywords{}
\def\myauthor{ALURU  AJAY}
\def\contact{19pa1a0410@vishnu.edu.in}
\def\mymodule{Future Wireless Comunnication (FWC22029)}
% #######################################
% #### YOU DON'T NEED TO TOUCH BELOW ####
% #######################################
\documentclass[10pt, a4paper]{article}
\usepackage[a4paper,outer=1.5cm,inner=1.5cm,top=1.75cm,bottom=1.5cm]{geometry}
\twocolumn
\usepackage{graphicx}
\graphicspath{{./images/}}
%colour our links, remove weird boxes
\usepackage[colorlinks,linkcolor={black},citecolor={blue!80!black},urlcolor={blue!80!black}]{hyperref}
%Stop indentation on new paragraphs
\usepackage[parfill]{parskip}
%% Arial-like font
\usepackage{lmodern}
\renewcommand*\familydefault{\sfdefault}
%Napier logo top right
\usepackage{watermark}
%Lorem Ipusm dolor please don't leave any in you final report ;)
\usepackage{karnaugh-map}
\usepackage{tabularx}
\usepackage{lipsum}
\usepackage{xcolor}
\usepackage{listings}
\usepackage{enumerate}
%give us the Capital H that we all know and love
\usepackage{float}
%tone down the line spacing after section titles
\usepackage{titlesec}
%Cool maths printing
\usepackage{amsmath}
%PseudoCode
\usepackage{algorithm2e}

\titlespacing{\subsection}{0pt}{\parskip}{-3pt}
\titlespacing{\subsubsection}{0pt}{\parskip}{-\parskip}
\titlespacing{\paragraph}{0pt}{\parskip}{\parskip}
\newcommand{\figuremacro}[5]{
    \begin{figure}[#1]
        \centering
        \includegraphics[width=#5\columnwidth]{#2}
        \caption[#3]{\textbf{#3}#4}
        \label{fig:#2}
    \end{figure}
}

\lstset{
frame=single, 
breaklines=true,
columns=fullflexible
}

\thiswatermark{\centering \put(1,-110.0){\includegraphics[scale=0.075]{iith.jpg}} 
 \put(400,-90){\includegraphics[scale=0.2]{wisig-full}} }

\title{\mytitle}
\author{\myauthor\hspace{1em}\\\contact\\IITH\hspace{0.5em}-\hspace{0.5em}\mymodule}
\date{}
\hypersetup{pdfauthor=\myauthor,pdftitle=\mytitle,pdfkeywords=\mykeywords}
\sloppy

% #######################################
% ########### START FROM HERE ###########
% #######################################
\begin{document}
   
	\maketitle
	\tableofcontents
	\begin{abstract}
	   The objective of this manual is to show how
       to Verify the Boolean Expression using \textbf{ARM} Processor-\textbf{VAMAN} Board\\
	        U’ + V = U’V’ + U’.V+U.V
	\end{abstract}
\section{Components}
\begin{tabularx}{0.45\textwidth} { 
  | >{\centering\arraybackslash}X 
  | >{\centering\arraybackslash}X
  | >{\centering\arraybackslash}X | }
\hline
\textbf{Component} & \textbf{Value} & \textbf{Quantity} \\      
\hline
Vaman Board &-& 1 \\
\hline 
Jumper wires&-&as required\\
\hline
\end{tabularx}
\begin{center}
    TABLE 1.0
\end{center}
\section{MATHEMATICAL PROOF}
To prove: U’ + V = U’V’ + U’.V+U.V
\vspace{0.3cm}\\
Using Distributive and Complement law
\vspace{0.2cm}\\
U'V' + U'V + UV = U’.(V+V’) + U.V
\vspace{0.1cm}\\
U'V' + U'V + UV = U’.(1) + U.V
\vspace{0.1cm}\\
U'V' + U'V + UV = U’+U.V
\vspace{0.1cm}\\
U'V' + U'V + UV = (U’+U). (U’+V)
\vspace{0.1cm}\\
U'V' + U'V + UV = 1.(U’+V)
\vspace{0.1cm}\\
U'V' + U'V + UV = U’+V
\section{Karunugh Map}
A Karnaugh map (K-map) is a visual method used to simplify the algebraic expressions in Boolean functions without having to resort to complex theorems or equation manipulations.  It involves fewer steps than the algebraic minimization technique to arrive at a simplified expression.\\
\newpage
Assign X=U'+V 
    \begin{center}
    \begin{karnaugh-map}[2][2][1][$V$][$U$]
        \minterms{0,1}
        \maxterms{2,3}
        \implicant{0}{1}
        %\implicant{1}{1}
        %\implicant{3}{3}
    \end{karnaugh-map}
        \end{center}
    
Assign Y=U'V'+U'V+UV
    \begin{center}
    \begin{karnaugh-map}[2][2][1][$V$][$U$]
        \minterms{0,1,3}
        \maxterms{2}
        \implicant{0}{0}
        \implicant{1}{1}
        \implicant{3}{3}
    \end{karnaugh-map} 
    \end{center}
    \begin{center}
    \end{center}
\section{Truth Table}
\hspace{2cm}
\\Above K-maps are verify using Table-0\\
Asumme that F=X=Y
    \begin{center}
\begin{tabularx}{0.4\textwidth} { 
  | >{\centering\arraybackslash}X 
  | >{\centering\arraybackslash}X 
  | >{\centering\arraybackslash}X
  | >{\centering\arraybackslash}X | }
\hline
\textbf{A} &\textbf{B} & \textbf{F} \\
\hline
0 & 0 & 1 \\  
\hline
0 & 1 & 1 \\ 
\hline
1 & 0 & 0 \\
\hline
1 & 1 & 1 \\
\hline
\end{tabularx}
\end{center}
\begin{center}
    TABLE 2.3
\end{center}
\section{Software}
\centering
1. Download the codes given in the link below and execute them.\\
\begin{table}[h]
\centering
\begin{tabular}{| c |} \hline
 \rule{0pt}{20pt} https://github.com/19PA1AO410/FWC-Module-1/
 \\blob/main/arm$\_$Assignment/src/main.c \\\hline
\end{tabular}
\end{table}
\section{Conclusion}
\begin{flushleft}
1. Distributive law is expressed by \\
\vspace{0.25cm}
U’ + V = U’V’ + U’.V+U.V with LHS = U’ + V, RHS = U’V’ + U’.V+U.V,\\
\vspace{0.25cm}
2. Codes are written for both Distributive laws and are executed using Vaman Board(Arm processor).
\vspace{0.2cm}\\
3. Result has been displayed on LEDs (i.e. LED1, LED2).
\vspace{0.2cm}\\
4. LED1 is assigned for LHS of the Boolean expression of Distributive Law.
\vspace{0.2cm}\\
5. LED2 is assigned for RHS of the Boolean expression of Distributive Law.
\vspace{0.2cm}\\
6. For random digital inputs U,and V as per Truth tables (at Vaman Board(Pigmy side)  pins 2,and 4), it has been noticed that, the output pins (18 and 21) of Vaman Board are at the same level.
\end{flushleft}
\end{document}



